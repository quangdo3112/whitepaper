% !TEX program = XeLaTeX
% !TEX encoding = UTF-8
\documentclass[UTF8,nofonts]{ctexart}
\setCJKmainfont[BoldFont=FandolSong-Bold.otf,ItalicFont=FandolKai-Regular.otf]{FandolSong-Regular.otf}
\setCJKsansfont[BoldFont=FandolHei-Bold.otf]{FandolHei-Regular.otf}
\setCJKmonofont{FandolFang-Regular.otf}

\usepackage{url}
\usepackage{cancel}
\usepackage{xspace}
\usepackage{graphicx}
\usepackage{multicol}
\usepackage{multirow}
\usepackage{subfig}
\usepackage{amsmath}
\usepackage{amssymb}
\usepackage[a4paper, width=186mm, top=18mm, bottom=18mm, includeheadfoot]{geometry}
\usepackage{booktabs}
\usepackage{array}
\usepackage{verbatim}
\usepackage{caption}
\usepackage{natbib}
\usepackage{booktabs}
\usepackage{float}
\usepackage{pdflscape}
\usepackage{mathtools}
\usepackage[usenames, dvipsnames]{xcolor}
\usepackage{afterpage}
\usepackage{pgf}
\usepackage{tikz}
\usepackage{dirtree}
\usepackage[style=american]{csquotes}
\usepackage{amsfonts}
\usepackage{tikz}
\usepackage{tkz-graph}
\usetikzlibrary{arrows,decorations.pathmorphing,automata,positioning,backgrounds,fit,shapes.symbols,chains,intersections}

\newtheorem{definition}{Definition}[section]
\newtheorem{theorem}{Theorem}[section]
\newtheorem{lemma}{Lemma}
\newtheorem{proof}{Proof} [section]

\usepackage[toc, page, title, titletoc, header]{appendix}
\usepackage{marginnote}
\usepackage{tablefootnote}

\renewcommand\abstractname{Abstract}


\usepackage{perpage} %the perpage package
\MakePerPage{footnote} %the perpage package command

\usetikzlibrary{shapes.geometric}%
\usepackage{color}
%\usepackage[pages=some, placement=top]{background}
\usepackage{eso-pic}
\usepackage[final]{pdfpages}

%\includepdf[pages=1]{cover}
\hyphenpenalty=750

\title{\textbf{Vite:}\textbf{一个异步高性能的通用去中心化应用平台}}
\author{
  刘春明\\
  \texttt{charles@vite.org}\\
  \and
  	王东\\
  	\texttt{daniel@loopring.org}\\
  	\and
  	伍鸣\\
  	\texttt{woo@vite.org}\\
 }

\makeatletter
\def\CTEX@section@format{\Large\bfseries}
\makeatother

\makeatletter
\newenvironment{tablehere}
 {\def\@captype{table}}
 {}
 
\newenvironment{figurehere}
 {\def\@captype{figure}}
 {}
\makeatother

\begin{document}

\maketitle

\begin{abstract}
摘要。
\end{abstract}

\begin{multicols}{2}

\section{简介}
\subsection{定义}

一个通用去中心化应用平台,可以支持一组智能合约,每个智能合约都是一个拥有独立状态以及不同操作逻辑的交易状态机,它们之间可以通过消息传递的方式进行通信。

从整体上,这个系统是一个交易状态机。系统的状态$\boldsymbol{s}\in\boldsymbol{S}$也称为世界状态,是由每一个独立账户的状态构成的。一个能够引起世界状态改变的事件称为交易。更形式化的定义如下:

\begin{definition}[交易状态机] \label{def-fsm}
一个交易状态机是一个四元组: $(\boldsymbol{T},\boldsymbol{S},\boldsymbol{g},\boldsymbol{{\delta}})$,其中,$\boldsymbol{T}$是交易的集合,$\boldsymbol{S}$是状态的集合,$\boldsymbol{g} \in \boldsymbol{S}$为初始状态,也称为“创世块”,$\boldsymbol{{\delta}:\boldsymbol{S}\times\boldsymbol{T}\rightarrow\boldsymbol{S}}$是状态转移函数。
\end{definition}

这种交易状态机的语义是一个离散迁移系统,定义如下:
\begin{definition}[交易状态机的语义] \label{def-semantics}
一个交易状态机$(\boldsymbol{T},\boldsymbol{S},\boldsymbol{s_{0}},\boldsymbol{{\delta}})$的语义是一个离散迁移系统: $(\boldsymbol{S},\boldsymbol{s_{0}},\boldsymbol{\rightarrow})$。$\boldsymbol{\rightarrow} \in \boldsymbol{S}\times\boldsymbol{S}$是迁移关系,
\end{definition}

同时,去中心化应用平台是一个分布式系统,具有最终一致性。通过某种共识算法,节点间可以就最终的状态达成一致。在现实场景中,智能合约的状态中保存的是一个去中心化应用的全部数据,体积比较大,无法在节点间传输。因此,节点间需要通过传递一组交易,来达成最终状态的一致。我们将这样的一组交易组织成某种特定的数据结构,通常称之为账本。

\begin{definition}[账本] \label{def-ledger}
账本是由一组交易构成的,具有递归构造的抽象数据类型,定义如下:
$$
\begin{cases}
 & l = \Gamma(S_{t})\\ 
 & l = l_{1} + l_{2}
\end{cases}
$$
其中,$S_{t} \in 2^{T}$,表示一组交易, $\Gamma \in 2^{T} \rightarrow L$,表示通过一组交易构造一个账本的函数,$L$是账本的集合,$\boldsymbol{+}: L \times L \rightarrow L$,表示将两个子账本合并成一个账本的操作。
\end{definition}

需要注意的是,在此类系统中,账本通常用来表示一组交易,而不是一个状态。账本有两个重要作用,一个是确定交易之间的顺序;另一个是防止交易被篡改。在以太坊\cite{wood2014ethereum}的协议中,账本是一个区块链结构,其中交易是全局有序的。修改账本中的一个交易,需要重新构造账本中的一个子账本,从而提高了篡改交易的成本。

根据同一组交易,可以构造出不同的账本,两个账本都是有效的,但它们所表示的交易顺序不同,因此可能会导致系统进入不同的状态。当这种情况发生时,我们称账本发生了“分叉”。
\begin{definition}[分叉] \label{def-fork}
$S_{t} \in 2^{T}$,$l_{1}=\Gamma_{1}(S_{t})$, $l_{2}=\Gamma_{2}(S_{t})$,若$l_{1} \neq l_{2}$,则称账本$l_{1}$和$l_{2}$是分叉的。 
\end{definition}

根据交易状态机的语义,我们可以很容易的证明,从一个初始状态开始,如果账本不分叉,则每个节点最终会进入相同的状态。那么,如果接收到分叉的账本,就一定进入不同的状态吗?这取决于账本中交易的内在逻辑,以及账本如何组织交易之间的偏序。现实中,经常会出现一些满足交换律的交易,却因为账本设计的问题,频繁的引起分叉。当系统从一个初始状态出发,接收两个分叉的账本,最终进入同一状态,我们称这两个账本为伪分叉账本。

\begin{definition}[伪分叉] \label{def-falsefork}
有初始状态$s_{0} \in S$,账本$l_{1},l_{2} \in L$,$s_{0}\overset{l_{1}}{\rightarrow}s_{1}, s_{0}\overset{l_{2}}{\rightarrow}s_{2}$。若$l_{1} \neq l_{2}$,且$s_{1}=s_{2}$,则称这两个账本$l_{1} ,l_{2}$为伪分叉(false fork)账本。
\end{definition}

一个设计良好的账本,应该尽量降低伪分叉发生的概率。

当分叉发生时,每个节点都需要从多个分叉的账本中选择一个,为确保状态的一致性,这些节点需要采用同一个算法完成选择,这个算法称为共识算法。

\begin{definition}[共识算法] \label{def-consensus}
共识算法是一个函数,它接收一个账本的集合,返回其中唯一一个账本:
$$\Phi: 2^{L} \rightarrow L$$
\end{definition}

共识算法是系统设计的一个重要内容,一个好的共识算法应该具有较高的相交速度(high convergence speed),减少共识在不同分叉间摇摆,并对恶意攻击具有较高的防范能力。

\subsection{当前进展}
\subsection{以太坊}
以太坊\cite{buterin2017ethereum}率先实现了这样一个系统。
在以太坊的设计中,世界状态的定义是:$S = \Sigma^{A}$,是由每个账户$a \in A$与该账户的状态$\sigma_{a} \in \Sigma$构成的映射。因此,以太坊的状态机中任何一个状态都是全局的,这表示一个节点在每个时刻都可以获取任何一个账户的状态。

以太坊的状态转移函数$\delta$,是由一组程序代码来定义的,每组代码被称为一个智能合约。以太坊定义了一个图灵完备的虚拟机,称为EVM,运行在其上的指令集称为EVM代码。用户可以通过一种语法类似于javascript的程序语言Solidity来开发智能合约,并编译成EVM代码,部署到以太坊上\cite{dannen2017introducing}。一旦智能合约部署成功,就相当于定义了该合约账户$a$收到一个交易后的状态转移函数$\delta_{a}$。
EVM目前在此类平台中被广泛使用,但它也存在一些问题。例如,缺少库函数支持,安全性问题突出等。

以太坊的账本结构是区块链\cite{nakamoto2008bitcoin},区块链由区块构成,每一个区块中包含一组交易的列表,后一个区块引用前一个区块的hash,构成一个链状结构。
\begin{equation}
\Gamma(\lbrace t_{1}, t_{2},...\vert t_{1}, t_{2},... \in T \rbrace) = (..., (t_{1}, t_{2},...)) 
\end{equation}
这个结构的最大好处是简单,但由于它维护的是所有交易的全序,任何两个交易交换顺序,都会生成一个新账本,也造成这种结构具有较高的分叉概率。事实上,在这个定义下,交易状态机的状态空间被看作一棵树:初始状态是根节点,不同的交易顺序代表不同的路径,叶子节点是最终状态。现实的情况下,大量叶子节点的状态是相同的,这就造成了大量的伪分叉。

以太坊的共识算法$\Phi$称为PoW,该算法率先在比特币协议中提出\cite{nakamoto2008bitcoin}。PoW算法依赖于某个易于验证但难于求解的数学问题,例如,根据一个hash函数$h:N \rightarrow N$,求解$x$,使$h(T+x) \geq d$,$d$是一个给定的数,称为难度,T是区块中包含的交易列表的二进制表示。在区块链的每个区块中,都会包含这类问题的一个解。将全部区块的难度加起来,就是一个区块链账本的总难度:
\begin{equation}
D(l) = D(\sum_{i} l_{i}) = \sum_{i} D(l_{i})
\end{equation}

因此,在从分叉中选择正确账本的时候,只要选择总难度最高的分叉即可:
\begin{equation}
 \Phi (l_{1}, l_{2}) = \begin{cases}
 & l_{1} \text{ if } D(l_{1}) \geqslant  D(l_{2})\\ 
 & l_{2} \text{ otherwise } 
\end{cases}
\end{equation}

PoW共识算法具有较好的安全性,迄今为止在比特币和以太坊中运行得很好。但这个算法有两个主要问题,第一是求解数学难题需要消耗大量计算资源,造成能源浪费;第二是该算法相交速度较慢,因而影响了系统整体的吞吐。目前,以太坊整体的TPS只有15左右,完全无法满足去中心化应用的需求。

\subsection{改进方向}
在以太坊诞生之后,以太社区和其他一些同类项目开始从不同方向对系统加以改进。从系统的抽象模型来看,可以改进的方向主要包括以下几个:
\begin{itemize}
	\item 改进系统状态$\boldsymbol{S}$	
	\item 改进状态迁移函数$\boldsymbol{\delta}$	
	\item 改进账本结构$\boldsymbol{\Gamma}$
	\item 改进共识算法$\boldsymbol{\Phi}$
\end{itemize}

\subsubsection{改进系统状态}
对系统状态的主要改进思路是将全局的世界状态局部化,每个节点不再关心全部交易和状态转移,只维护整个状态机的一个子集。这样集合$S$和集合$T$的势都大为缩减,从而提高了系统扩展性。此类系统包括:Cosmos\cite{cosmos},Aelf\cite{aelf},PChain等。

从本质上讲,此类基于侧链的方案牺牲了状态的全局性,以换取系统的扩展性。这使得每个运行在其上的dApp的去中心化程度都被削弱——一个智能合约的交易历史不再被全网每一个节点保存,而只被一部分节点保存。
除此之外,跨合约交互也会成为此类系统的瓶颈。例如,Cosmos中,不同的Zone交互,需要通过一个共同的链Hub来完成\cite{cosmos}。

\subsubsection{改进状态迁移函数}
一些项目立足于改进EVM,提供更为丰富的智能合约编程语言。例如,RChain	定义了一种基于$\pi$演算的智能合约语言Rholang,NEO中的智能合约称为NeoContract,可以用Java,C\# 等主流编程语言开发。

\subsubsection{改进账本结构}
账本结构的改进方向是构造等价类,将多个交易全局有序的线性账本规约为一个只记录部分偏序关系的非线性账本,这种非线性账本结构是一个DAG(有向无环图)。目前,Byteball\cite{byteball},IOTA,Nano等项目基于DAG的账本结构实现了加密货币功能。也有一些项目在尝试利用DAG实现智能合约,但迄今为止在这个方向上的改进还在探索中。

\subsubsection{改进共识算法}
共识算法的改进大部分是为了提高系统的吞吐,主要方向是抑制伪分叉的产生。下面我们讨论伪分叉与哪些因素有关。

\includegraphics[width=.8\linewidth]{image/falsefork.png}

如图所示,$L$是针对某个交易集合所有可能的分叉账本的集合,$S$是以不同顺序执行这一组交易,所能到达的状态的集合。根据定义 \ref{def-fork},映射$f: L \rightarrow S$是一个满射;而根据定义 \ref{def-falsefork},这个映射不是单射。下面我们来计算伪分叉的概率:

假设共有$C$个用户有权生产账本,$M = \vert L \vert$,$N = \vert S \vert$,$M_{i} = \vert L_{i} \vert$, 其中,$L_{i} = \lbrace l \vert f(l) = s_{i}, s_{i} \in S \rbrace$。
则伪分叉概率为:
\begin{equation}
P_{ff} = \sum_{i=1}^{N}\left ( \frac{M_{i}}{M} \right )^{C} - \frac{1}{M^{C-1}}
\end{equation}

从这个公式可以看出,为了降低伪分叉概率,可以有两种途径:
\begin{itemize}
	\item 在账本集合$L$上建立等价关系,对其划分等价类,构造分叉更少的账本
	\item 限制有权生产账本的用户,从而减少$C$
\end{itemize}

第一种途径是Vite设计的重要方向,后文将详细论述;第二种途径现在已被多种算法所采用。在PoW算法中,任何用户都有权生产区块;而PoS算法将生产区块的权力限制在那些拥有系统权益的用户中;DPoS算法将有权生产区块的用户进一步限制在一组代理节点范围内。
	
目前,通过改良共识算法,已经产生出一些比较有影响力的项目。例如,Cardano采用了一种PoS算法,称为Ouroboros,文献\cite{bernardo2017ouroboros}对该算法相关性质给出了严格证明;EOS采用了DPoS算法,通过快速生产区块,提高系统的吞吐;Qtum\cite{dai2017smart}的共识算法也是一种PoS算法;RChain采用的Casper算法也是一种PoS算法。

还有一些其他项目在共识算法的改进上提出了自己的方案。NEO采用了一种BFT算法,称为dBFT;Cosmos采用了一种称为Tendermint\cite{tendermint}的算法。

\section{账本结构}


\section{共识算法}
\subsection{交易确认}

\section{限流}
$$
TPS_{n} =  \frac{10}{S\cdot(H_{n}  -  H_{n-9} + 1)}
$$

\section{虚拟机}

\section{总结}

Vite的特点总结如下:

\begin{itemize}
	\item 特点一。
\end{itemize}

\section{致谢}
致谢部分。

\end{multicols}

\bibliography{vite}
\bibliographystyle{unsrt}

\end{document}
